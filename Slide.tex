\documentclass[11pt]{beamer}
\usetheme{CambridgeUS}
\setbeamertemplate{footline}
{%
	\leavevmode%
	\hbox{%
		\begin{beamercolorbox}[wd=.4\paperwidth,ht=2.5ex,dp=1.125ex,center]{author in head/foot}%
			\usebeamerfont{author in head/foot}\insertshortauthor
		\end{beamercolorbox}%
		\begin{beamercolorbox}[wd=.4\paperwidth,ht=2.5ex,dp=1.125ex,center]{title in head/foot}%
			\usebeamerfont{title in head/foot}\insertshorttitle
		\end{beamercolorbox}%
		\begin{beamercolorbox}[wd=.2\paperwidth,ht=2.5ex,dp=1.125ex,center]{date in head/foot}%
			\usebeamerfont{date in head/foot}\insertframenumber{} / \inserttotalframenumber\hspace*{2ex}
		\end{beamercolorbox}%
	}%
	\vskip0pt%
}



%-----------------------------------------------------------------------
% Packages
\usepackage[utf8]{inputenc}
\usepackage{amsmath}
\usepackage{amsfonts}
\usepackage{amssymb}
\usepackage{graphicx}

%-----------------------------------------------------------------------
% Package customization
\definecolor{links}{HTML}{FC0D30}
\hypersetup{colorlinks,linkcolor=,urlcolor=links}

%-----------------------------------------------------------------------
% Commands
\renewcommand{\emph}[1]{\textbf{#1}}
\graphicspath{ {immagini/} {images/} }

%-----------------------------------------------------------------------
% Headings
\author{Matteo Mazzaretto}
\title{Presentazione università}
\setbeamertemplate{navigation symbols}{}
\setbeamertemplate{date}{}
\logo{%
	\makebox[0.95\paperwidth]{
		\href{mailto: matteovery498@gmail.com}{\includegraphics[width=0.05\textwidth]{logo_gmail}} \hfill
		\href{https://www.instagram.com/matteomazzaretto/}{\includegraphics[width=0.05\textwidth]{logo_instagram}}
	}%
}


%-----------------------------------------------------------------------
% Document
\begin{document}
	
	\begin{frame}
		\titlepage
	\end{frame}
	
	\begin{frame}{Disclaimer}
		Come esempio prendo l'Università di Padova, ovvero l'università che frequento.\\
		Ogni università ha la sua organizzazione, quanto segue ha quindi solo scopo rappresentativo per farsi un'idea e non rappresenta tutte le università.
	\end{frame}
	
	\begin{frame}{Pro}
		\begin{minipage}[t]{0.5\textwidth}
			\raggedright
			\vspace{0pt} % forza l'allineamento in alto
			\begin{center}
				\begin{itemize}
					\item Frequenza non obbligatoria
					\item Totale autonomia
					\item I corsi durano 3 mesi e mezzo
					\item Conoscenze molto più approfondite
				\end{itemize}
			\end{center}
		\end{minipage}
		\begin{minipage}[t]{0.4\textwidth}
			\centering
			\vspace{0pt} % forza l'allineamento in alto
			\includegraphics[width=\textwidth]{happy}
		\end{minipage}
	\end{frame}
	
	\begin{frame}{Contro}
		\begin{minipage}[t]{0.5\textwidth}
			\raggedright
			\vspace{0pt} % forza l'allineamento in alto
			\begin{itemize}
				\item Studio molto più complesso
				\item Totale autonomia
				\item UniPd che spamma mail
				\item Orari non sempre comodi
			\end{itemize}
		\end{minipage}
		\begin{minipage}[t]{0.4\textwidth}
			\centering
			\vspace{0pt} % forza l'allineamento in alto
			\includegraphics[width=\textwidth]{angry}
		\end{minipage}
	\end{frame}
	
	\begin{frame}{Organizzazione lezioni}
		Prendendo come esempio l'anno accademico UniPd 2025-26:\newline \newline
		\begin{tabular}{p{0.3\textwidth} p{0.35\textwidth} p{0.35\textwidth}}
			Periodo & Inizio & Fine \\
			& & \\
			Primo semestre & 29 Settembre 2025 & 17 Gennaio 2026 \\
			& & \\
			Secondo semestre & 23 Febbraio 2026 & 12 Giugno 2026 \\
		\end{tabular}
	\end{frame}
	
	\begin{frame}{Organizzazione sessioni}
		Prendendo come esempio l'anno accademico UniPd 2025-26:\newline \newline
		\begin{tabular}{p{0.3\textwidth} p{0.35\textwidth} p{0.35\textwidth}}
			Periodo & Inizio & Fine \\
			& & \\
			Sessione invernale & 19 Gennaio 2026 & 21 Febbraio 2026 \\
			& & \\
			Sessione estiva & 15 Giugno 2026 & 18 Luglio 2026 \\
			& & \\
			Sessione di recupero & 19 Agosto 2026 & 19 Settembre 2026 \\
		\end{tabular}
	\end{frame}
	
	\begin{frame}{Organizzazione sessioni: parte 2}
		Ogni corso ha 5 appelli durante tutto l'anno accademico.\\
		I corsi del primo semestre hanno due appelli nella sessione invernale, due appelli nella sessione estiva e un appello nella sessione di recupero.\\
		I corsi del secondo semestre hanno due appelli nella sessione estiva, due nella sessione di recupero e uno nella sessione invernale.\\
		Solitamente queste sono le organizzazioni ma possono cambiare da corso a corso.
	\end{frame}
	
	\begin{frame}{Organizzazione esami: modalità d'esame}
		Le modalità d'esame principali sono:
		\begin{itemize}
			\item Scritto
			\item Orale
			\item Progetto + Scritto
			\item Scritto + Orale (solitamente in date diverse)
		\end{itemize}
		Ogni professore a propria discrezione può decidere di svolgere anche gli esami parziali.\\
		Ogni esame ha un voto da 0 a 30 e lode. \\
		Per essere considerato sufficiente serve un voto $>=$18.
	\end{frame}
	
	\begin{frame}
		\begin{minipage}[t]{0.45\textwidth}
			\includegraphics[width=0.45\paperwidth,height=0.45\paperheight]{IsThisACFU}
		\end{minipage}
		\hspace{0.02\textwidth}
		\begin{minipage}[t]{0.45\textwidth}
			\includegraphics[width=0.45\paperwidth,height=0.45\paperheight]{Computer_CFU}
		\end{minipage}
	\end{frame}
	
	\begin{frame}{Organizzazione esami: CFU}
		Ogni esame vale un numero di CFU (Crediti Formativi Universitari), i quali servono a quantificare lo studio necessario da dedicare ad ogni esame.\\
		Un CFU corrisponde a 25 ore di lavoro (distribuito fra lezione e studio personale).\\
		È un identificatore abbastanza vago, poiché ogni esame può cambiare da persona a persona, sia in termini di studio che complessità.\\
		Servono per calcolare la media ponderata.
	\end{frame}
	
	\begin{frame}{Valore laurea}
		In Italia, ogni laurea triennale necessita 180 CFU per l'acquisizione, invece la laurea magistrale a ciclo unico richiede 300 CFU.\\
		Infatti, l'ideale è ottenere 60 CFU all'anno per laurearsi in tempo.
	\end{frame}
	
	\begin{frame}{Calcolo media ponderata}
		Nella media ponderata ogni voto conta di più o di meno in base ai suoi CFU.\\
		Si moltiplica ogni voto per i suoi CFU, si somma tutto, e si divide per la somma dei CFU.\\
		Essendo un po' complicato a parole, cerco di spiegare con un facile esempio.
	\end{frame}
	
	\begin{frame}{Esempio di calcolo}
		Ipotizziamo di avere i seguenti 2 esami:
		\begin{itemize}
			\item Informatica, 3 CFU, 30/30
			\item Microeconomia, 12 CFU, 18/30
		\end{itemize}
		La media ponderata è data da
		\begin{center}
			$\frac{(3*30)+(12*18)}{(3+12)}=$ 20.4
		\end{center}
	\end{frame}
	
	\begin{frame}{Base di voto di laurea}
		Il voto di laurea parte da 66 fino ad un massimo di 110 e lode.\\
		Ogni corso di laurea ha diversi bonus al momento della laurea, ad esempio il corso di Informatica prevede un bonus di 2 punti se ti laurei entro settembre del terzo anno.\\
		Per avere il voto di base si fa il seguente calcolo:\\
		\begin{center}
			Media ponderata * 110 / 30
		\end{center}
		Relativo all'esempio precedente:
		\begin{center}
			$\frac{20.4 * 110}{30}$ = 74.8
		\end{center}
		Comunque non preoccupatevi, l'app UniPd fa tutto automaticamente, ma è giusto che sappiate come funziona!
	\end{frame}
		
	\begin{frame}{Sito UniPd}
		\includegraphics[width=\paperwidth,height=\paperheight]{sito_unipd}
	\end{frame}
	
	\begin{frame}{Come cercare il corso che si vuole}
		\begin{minipage}[t]{0.45\textwidth}
			\begin{center}
				1
			\end{center}
			\includegraphics[width=0.45\paperwidth,height=0.45\paperheight]{sito_unipd_2}
		\end{minipage}
		\begin{minipage}[t]{0.45\textwidth}
			\begin{center}
				2
			\end{center}
			\includegraphics[width=0.45\paperwidth,height=0.45\paperheight]{trova_corso}
		\end{minipage}
	\end{frame}
	
	\begin{frame}{Come cercare il corso che si vuole: parte 2}
		\includegraphics[width=\paperwidth,height=\paperheight]{research}
	\end{frame}
	
	\begin{frame}{Istruzioni per l'iscrizione}
		\includegraphics[width=\paperwidth,height=0.8\paperheight]{ammissione}
	\end{frame}
	
	\begin{frame}{Istruzioni per l'iscrizione: parte 2}
		\includegraphics[width=\paperwidth,height=0.8\paperheight]{ammissione_2}
	\end{frame}
	
	\begin{frame}{Istruzioni per l'iscrizione: parte 3}
		Per l'ammissione, e in generale per avere una carriera in un corso di laurea di UniPd, serve avere un account su Uniweb (il link si trova facilmente nella pagina principale).\\
		Ogni persona può effettuare la preimmatricolazione da lì disponendo 2 possibili scelte di cui la prima ha priorità maggiore.\\
	\end{frame}
	
	\begin{frame}{Discussione istruzioni per l'iscrizione}
		\href{https://www.unipd.it/sites/unipd.it/files/2025/2025_Bando_L_Scienze-programmati_V3.pdf}{Bando di ammissione Informatica}
	\end{frame}
	
	\begin{frame}{TOLC}
		Il TOLC è il test per l'iscrizione alle università in Italia.\\
		Esistono diversi tipi per accedere a diverse facoltà.
		\begin{itemize}
			\item TOLC-I: ingegneria, informatica
			\item TOLC-E: economia
			\item altri (definito nelle istruzioni per l'iscrizione)
		\end{itemize}
		Comunque, ogni Università ha la propria validità per il TOLC, ad esempio Informatica per l'Università di Udine richiede il TOLC-S.\\
		Per informarsi, bisogna visitare il sito \href{https://cisiaonline.it/}{CISIA}.\\
		In quel sito ti puoi esercitare, capire come funziona, ma soprattutto puoi prenotare il test.\\
		Anche nelle lauree ad accesso libero, è necessario aver fatto almeno un TOLC presso la piattaforma Cisia.\\
	\end{frame}
	
	\begin{frame}{TOLC: parte 2}
		\includegraphics[width=\paperwidth,height=0.8\paperheight]{datetolc}
	\end{frame}
	
	\begin{frame}{TOLC: parte 3}
		Il TOLC è un esame che puoi affrontare sia in presenza che da casa.\\
		Puoi farlo con qualsiasi università, la sua validità resta in tutta Italia ed è valido per 1 o 2 anni (questo dipende dal corso e/o dall'università).\\
		Per sostenerlo bisogna pagare 35 euro.\\
		Online si trovano tranquillamente i libri di esercitazioni, suggerisco solo quello teoria+esercizi.\\
		È molto conveniente farlo bene, poiché è il criterio più importante per l'accesso ai corsi, inoltre se fatto male porta all'OFA (Obbligo Formativo Aggiuntivo).\\
		La guida completa è in questo \href{https://guide.cisiaonline.it/}{link}, vi suggerisco di cercare attentamente ciò che è necessario oppure di scrivermi per qualche informazione in più!
	\end{frame}
	
	\begin{frame}{\href{https://didattica.unipd.it}{didattica.unipd.it}}
		\includegraphics[width=\paperwidth,height=0.8\paperheight]{didattica}
	\end{frame}
	
	\begin{frame}{Dove trovare le informazioni importanti}
		Per trovare le informazioni importanti bisogna:
		\begin{enumerate}
			\item Cliccare sull'anno di cui ci si vuole informare
			\item Cliccare sulla scuola di propria preferenza (schiacciare \textbf{Corsi di Laurea Magistrale a Ciclo Unico} se ci si vuole informare su quelle, per esempio Giurisprudenza)
			\item Cercare il corso di proprio gradimento
			\item Seguire la slide successiva
		\end{enumerate}
	\end{frame}
	
	\begin{frame}{Allegato, percorso formativo}
		\includegraphics[width=\paperwidth,height=0.8\paperheight]{allegato_percorso}
	\end{frame}
	
	\begin{frame}{Discussione allegato e percorso formativo}
		Segue una discussione sull'allegato e sul percorso formativo, in modo da illustrarne le varie componenti.
	\end{frame}
	
	\begin{frame}{Propedeuticità}
		All'università, una propedeuticità indica un corso, attività o requisito che deve essere soddisfatto obbligatoriamente prima di poter sostenere l'esame di uno o più insegnamenti specifici.\\
		Ad esempio, per il corso \textbf{Programmazione ad Oggetti}, è obbligatorio aver passato il corso di \textbf{Programmazione}.
	\end{frame}
	
	\begin{frame}{Informazioni singolo corso}
		Nella pagina in cui si trovano l'allegato e il percorso formativo, si possono trovare informazioni relative al corso singolo. \\
		Basta schiacciare su \textit{Attuale A.A. 2025/26} per avere informazioni più recenti relativamente ai corsi.
	\end{frame}
	
	\begin{frame}{Moodle}
		Il Moodle dell'università si trova al seguente \href{https://www.unipd.it/elearning}{link} scegliendo quello della propria area dentro la sezione \textbf{Piattaforme Moodle delle scuole}.\\
		Per economia ed informatica si usa quello della macroarea STEM (Science, Technology, Economy and Engineering).\\
		Una volta selezionato il proprio, si fa il login, successivamente:
	\end{frame}
	
	\begin{frame}{Moodle: parte 2}
		\begin{center}
			\includegraphics[width=0.7\paperwidth,height=0.45\paperheight]{moodle_1}
		\end{center}
		Lì trovate il QR code necessario per collegare l'applicazione del telefono al Moodle corretto.
	\end{frame}
	
	\begin{frame}{Applicazioni utili}
		\begin{center}
			\includegraphics[width=0.5\paperwidth,height=0.8\paperheight]{apputili}
		\end{center}
	\end{frame}
	
	\begin{frame}{Contribuzione e borse di studio}
		L'importo delle tasse per l'iscrizione si basa principalmente sull'ISEE, un indicatore socio-economico che permette di misurare e confrontare la condizione economica delle famiglie.\\
		Tutte le indicazioni necessarie sono in questo \href{https://www.unipd.it/contribuzione-studentesca}{link}.\\
		Qui puoi trovare l'elenco di tutte le borse di studio disponibili, il simulatore delle tasse, il bando contribuzione ed esoneri (anche per gli studenti lavoratori con reddito annuo lordo di almeno 3500 euro) e anche i bandi per collaborazioni di studenti e studentesse.
	\end{frame}
	
	\begin{frame}{Altre cose utili}
		\begin{itemize}
			\item Corso singolo
			\item Erasmus
			\item Distributore d'acqua
			\item Mensa
			\item Aule studio
			\item Tutorati
		\end{itemize}
	\end{frame}
	
	\begin{frame}
		L'Università è piena di cose da scoprire col tempo, sia belle che brutte.\newline \newline
		Purtroppo è impossibile elencare e spiegare tutto nel minimo dettaglio.\newline \newline
		Ad ogni modo, vi auguro il meglio per il futuro :)
	\end{frame}
	
	\begin{frame}{Conclusione}
		Per ogni domanda:
		\begin{itemize}
			\item email: \href{mailto:matteoevery498@gmail.com}{matteoevery498@gmail.com}
			\item instagram:
			\href{https://www.instagram.com/matteomazzaretto/}{matteomazzaretto}
			\item per aiutarmi:
		\end{itemize}
		
		\begin{columns}
			\begin{column}{0.15\textwidth}
				\includegraphics[width=2.00\textwidth]{qr}
			\end{column}
		\end{columns}
	\end{frame}
	
\end{document}
